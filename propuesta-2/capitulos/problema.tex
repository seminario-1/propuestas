Se carece de modelos de rendimiento que contribuyan a caracterizar el comportamiento de funciones en la nube alojadas en plataformas FaaS bajo distintas cargas de trabajo. Contar con tales modles permitiría validar si las funciones en la nube pueden cumplir criterios de calidad de servicio especificados.

%para de esta forma tener la capacidad de validar si estas pueden llegar a cumplir los criterios de calidad de servicio establecidos.

En las plataformas FaaS en las que se ejecutan funciones en la nube, la infraestructura tecnológica subyacente se oculta por completo de los desarrolladores y diseñadores. El conocimiento de la influencia de esta infraestructura y su configuración es vital para que los arquitectos de software puedan obtener predicciones significativas del comportamiento de una función pues, al omitirse la influencia que esta tiene, puede conducir a la generación de predicciones erróneas con respecto del rendimiento de una función. Una función que reporte tiempos de respuesta sumamente prolongados o bien la utilización de significativas cantidades de recursos puede generar grandes costos económicos y hasta llegar a ser rechazada por la plataforma FaaS.

%Las decisiones que hayan sido tomadas con poca información durante las etapas de diseño usualmente son muy difíciles de alterar y pueden impactar de forma negativa en los niveles requeridos de rendimiento de un sistema una vez que este ha sido puesto en producción. Es por esto que los arquitectos y diseñadores necesitan contar con la habilidad de predecir el rendimiento de una función trabajando a partir de diseños abstractos sin tener acceso a la implementación completa de la aplicación.