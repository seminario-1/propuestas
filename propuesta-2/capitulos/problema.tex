En las plataformas FaaS en las que se ejecutan de funciones en la nube, la infraestructura tecnológica subyacente se oculta por completo de los desarrolladores y diseñadores. El conocimiento de la influencia de esta infraestructura y su configuración es vital para los arquitectos de software para obtener predicciones significativas del comportamiento de una función y, al omitirse la influencia que esta tiene, se puede conducir a la generación de predicciones erróneas con respecto al rendimiento de una función. Una función que reporte tiempos de respuesta sumamente prolongados o bien la utilización de grandes cantidades de recursos puede generar grandes costos económicos y hasta llegar a ser rechazada por la plataforma de FaaS.

Las decisiones que han sido tomadas con poca información durante las etapas de diseño usualmente son muy difíciles de alterar y pueden impactar de forma negativa en los niveles requeridos de rendimiento de un sistema una vez que este ha sido puesto en producción. Es por esto que los arquitectos y diseñadores necesitan contar con la habilidad de predecir el rendimiento de una función trabajando a partir de diseños abstractos sin tener acceso a la implementación completa de la aplicación.