\section{Innovación}
Los aspectos más novedosos que se aportarán esta tesis serán:
\begin{enumerate}
    \item Dar a conocer un método mediante el cual se pueden obtener estimaciones del rendimiento de una función en la nube
    \item El uso modelado y simulación basado en componentes para modelar el rendimiento de funciones en la nube en plataformas FaaS
    \item Proporcionar, a partir de lo anterior, un modelo del rendimiento de una función en la nube    
\end{enumerate}

Con respecto a (1), en \cite{vanEyk:2018:SRC:3185768.3186308} se reporta que la predicción del rendimiento es uno de los principales retos de investigación a en plataformas FaaS y que no se cuentan con modelos de rendimiento que tomen en cuenta las características de FaaS. Esto abre la posibilidad de explorar la aplicabilidad de técnicas conocidas de modelaje de rendimiento en sistemas de software. El modelado y simulación de arquitecturas de software basadas en componentes (2), representa una alternativa atractiva para abordar este problema, esto porque es un enfoque en donde hay una comunidad de investigación y desarrollo activa que ha logrado generar diferentes tipos de herramientas para la estimación del rendimiento de un sistema. 

La obtención de un modelo de rendimiento de una función en la nube (3), representaría un aporte relevante para SPE y FaaS porque significaría que existe una forma por medio de la cual evaluar el comportamiento de una función sin que esta esté necesariamente instalada en una plataforma FaaS y además haría que cambios futuros que necesite esa función puedan ser modelados con anterioridad, simularlos y obtener predicciones del impacto de los cambios.



\section{Impacto}
En distintos reportes sobre el estado de tecnologías \emph{serverless} en la industria \cite{dzone-cloud-2018,rightscale-2018,digital-ocean-2018,pivotal-june-2018} se señala que la adopción de este tipo de tecnologías va en franco aumento. En \cite{rightscale-2018} se reporta una tasa de crecimiento del 25\% en su adopción con respecto al 2017. Cloud Foundry Foundation \cite{pivotal-june-2018} indica que en 2017 la mayoría de sus encuestados no estaban usando tecnologías \emph{serverless} mientras que para el 2018, solamente 43\% no lo estaba haciendo. En el informe anual del estado de tecnologías en la nube de DZone\cite{dzone-cloud-2018} se notó un incremento del 14\% con respecto al 2017 en el uso de tecnologías \emph{serverless}.

Pese a que los niveles de adopción de esta tecnología van en aumento, estos mismos reportes señalan incovenientes tales como:
\begin{enumerate}
    \item Se tiene que depender de los niveles de servicio un proveedor
    \item Dificultad para monitorear y \emph{debuggear}
    \item Preocupación por parte de los desarrolladores por el ``cómo funciona'' la función en la plataforma FaaS: ¿se estarán asignando los recursos adecuados para mi función en la plataforma FaaS? ¿Cómo y cuándo se hace?
    \item Límites de tiempo de espera: dependiendo del tiempo de ejecución una función en la nube podría llegar a ser cancelada por la plataforma FaaS
\end{enumerate}

Lo anterior refleja que aún existe una especie de ``área gris'' alrededor del uso de las tecnologías \emph{serverless} y en particular funciones en la nube. Esto es lo que van Eyk et al.\cite{vanEyk:2018:SRC:3185768.3186308} llama brecha de la información: el usuario de FaaS no está consciente de los recursos de hardware en los que las funciones son ejecutadas, mientras que por otro lado, la plataforma de FaaS no tiene información acerca de los detalles de la implementación de la función.

Analizar el rendimiento de una función en la nube y obtener un modelo de este contribuiría a tener un mejor entendimiento de esta tecnología y de cómo esta es gestionada por la plataforma FaaS. Esto permitiría a los arquitectos y diseñadores tener mayor control sobre los cambios en una función para que esta no solamente pueda cumplir con los requerimientos de calidad de servicio sino que también al ser \emph{serverless} un servicio que se cobra por demanda colaboraría a reducir gastos, ya que una función que tenga un tiempo de ejecución menor generá menores costos en la plataforma FaaS.

\section{Profundidad}
Para lograr obtener un modelo de rendimiento de una función en la nube se plantean las siguientes actividades:
\begin{itemize}
    \item Implementación de un caso de uso de referencia de una función en la nube 
    \item Obtención un modelo:
    \begin{itemize}
        \item Realizar pruebas de carga sobre la función en la nube seleccionada
        \item Obtener métricas asociadas al rendimiento desde la bitácora de ejecución
        \item Utilizar las métricas obtenidas como entrada para una herramienta de extracción de modelos de rendimiento. La herramienta generará como resultado un modelo de rendimiento
    \end{itemize}
    \item Ejecutar simulaciones sobre el modelo obtenido con el fin de verificar si los resultados obtenidos logran estimaciones adecuadas de la ejecución de la función o si por el contrario se necesita calibrar el modelo.
\end{itemize}

\subsection{Implementación de un caso de uso de referencia de una función en la nube} 
Se seleccionará un caso de uso en el cual el uso de una función en la nube haya mostrado ser una solución adecuada. Una vez identificado este caso de uso, se implementará la función y se instalará en una plataforma FaaS.

\subsection{Obtención de un modelo} 
Las plataformas FaaS proporcionan bitácoras en donde se registran métricas asociadas a la ejecución de la función en la nube. Se planea entonces ejecutar pruebas sobre la función en la nube bajo diferentes cargas de trabajo y de estar forma poder contar con una bitácora(s) con todas las métricas obtenidas durante el tiempo que tomó la ejecución de las pruebas. 


